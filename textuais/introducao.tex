\chapter[Introdução]{Introdução}

Na introdução deve-se fazer a contextualização da pesquisa,apresentando o tema, o problema a ser abordado, a(s) hipótese(s) ou pressupostos e a justificativa. O tema é uma delimitação do assunto da pesquisa, a qual pode ser relacionada à realidade do pesquisador tendo em vista sua intenção de conhecer melhor um assunto, investigá-lo ou realizar algo de maneira mais eficiente em relação ao mesmo. A justificativa reflete o “porquê” da realização da pesquisa, buscando identificar os motivos da preferência pelo tema escolhido e sua importância em comparação a outros temas. O conteúdo de uma justificativa deve ser constituído de dois aspectos: relevância (social,científica ou acadêmica) do tema e abrangência do assunto. Eu adiciono coisas aqui e aparece no lado direito